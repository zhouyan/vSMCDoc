\section{Introduction}
\label{sec:Introduction}

Sequential Monte Carlo (\smc) methods are a class of sampling algorithms that
combine importance sampling and resampling. They have been primarily used as
``particle filters'' to solve optimal filtering problems; see, for example,
\textcite{Cappe:2007hz} and \textcite{Doucet:2011us} for recent reviews. They
are also used in a static setting where a target distribution is of interest,
for example, for the purpose of Bayesian modeling. This was proposed by
\textcite{DelMoral:2006hc} and developed by \textcite{Peters:2005wh} and
\textcite{DelMoral:2006wv}. This framework involves the construction of a
sequence of artificial distributions on spaces of increasing dimensions which
admit the distributions of interest as particular marginals.

\smc algorithms are perceived as being difficult to implement while general
tools were not available until the development of \textcite{Johansen:2009wd},
which provided a general framework for implementing \smc algorithms. \smc
algorithms admit natural and scalable parallelization. However, there are only
parallel implementations of \smc algorithms for many problem specific
applications, usually associated with specific \smc related researches.
\textcite{Lee:2010fm} studied the parallelization of \smc algorithms on \gpu{}s
with some generality. There are few general tools to implement \smc algorithms
on parallel hardware though multicore \cpu{}s are very common today and
computing on specialized hardware such as \gpu{}s are more and more popular.

The purpose of the current work is to provide a general framework for
implementing \smc algorithms on both sequential and parallel hardware. There
are two main goals of the presented framework. The first is reusability. It
will be demonstrated that the same implementation source code can be used to
build a serialized sampler, or using different programming models (for example,
OpenMP and Intel Threading Building Blocks) to build parallelized samplers for
multicore \cpu{}s. The second is extensibility. It is possible to write a
backend for \vsmc to use new parallel programming models while reusing existing
implementations. It is also possible to enhance the library to improve
performance for specific applications. Almost all components of the library can
be reimplemented by users and thus if the default implementation is not
suitable for a specific application, they can be replaced while being
integrated with other components seamlessly.

Section~\ref{sec:Sequential Monte Carlo} introduces the \smc algorithm and
notations used throughout this guide. Section~\ref{sec:Basic usage} introduces
the basic features of the library. It is followed by section~\ref{sec:Advanced
  usage}, which details a few more advanced usage of the library.
Section~\ref{sec:Configuration macros} to~\ref{sec:Utilities} introduces some
useful additional features. It is not practical and of little interest to the
users to include all the details of the library in this guide. A
Doxygen\footnote{\url{http://www.stack.nl/~dimitri/doxygen/}} generated
reference manual can be access
online\footnote{\url{http://zhouyan.github.io/vSMCDoc/\version/}}. Last,
appendix~\appref{sec:Source code of complete programs} shows the complete
source code of programs discussed in the following guide.
