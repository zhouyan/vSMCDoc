\section{Resampling}
\label{sec:Resampling}

The library supports resampling in a more general way than the algorithm
described in section~\ref{sec:Sequential Monte Carlo}. Recall that, given a
particle system $\{W^{(i)},X^{(i)}\}_{i=1}^N$, a new system $\{\bar{W}^{(i)},
\bar{X}^{(i)}\}_{i=1}^M$ is generated. Regardless other statistical properties,
in practice, such an algorithm can be decomposed into three steps. First, a
vector $\{r_i\}_{i=1}^N$ is generated such that $\sum_{i=1}^N r_i = M$. Then a
vector $\{a_i\}_{i=1}^M$ is generated such that, $\sum_{i=1}^M
\mathbb{I}_{\{j\}}(a_i) = r_j$. And last, set $\bar{X}^{(i)} = X^{(a_i)}$.

The first step determines the statistical properties of the resampling
algorithm. The library defines all algorithms discussed in
\textcite{Douc:2005wa}. Samplers can be constructed with builtin schemes as
seen in section~\ref{ssub:Implementations}. In addition, samplers can also be
constructed with user defined resampling operations. Below is the signature,
\begin{cppcode}
  template <typename IntType, typename RNGType>
  void resample(std::size_t M, std::size_t N, RNGType &rng,
  const double *weight, IntType *replication);
\end{cppcode}
The last parameter is the output vector $\{r_i\}_{i=1}^N$. The builtin schemes
are implemented as classes with \cppinline{operator()} conforms to the above
signature. For example, \cppinline{ResampleMultinomial} implements the
multinomial resampling algorithm.

To transform $\{r_i\}_{i=1}^N$ into $\{a_i\}_{i=1}^M$, one can call the
following function,
\begin{cppcode}
  template <typename IntType1, typename IntType2>
  void resample_trans_rep_index(std::size_t M, std::size_t N,
  const IntType1 *replication, IntType2 *index);
\end{cppcode}
where the last parameter is the output vector $\{a_i\}_{i=1}^M$. This function
guarantees that $a_i = i$ if $r_i > 0$. However, its output may not be optimal
for all applications. The last step of a resampling operation, the copying of
particles can be the most time consuming one, especially on distributed
systems. The topology of the system will need to be taking into consideration
to achieve optimal performance. In those situations, it is best to use
\cppinline{ResampleMultinomial} etc., to generate the replication numbers, and
manually perform the rest of the resampling algorithm.

\subsection{Resizing a sampler}
\label{sub:Resizing a sampler}

Now, we provide an example of changing sampler size,
\begin{cppcode}
  // sampler is an existing Sampler<T> object
  auto N = sampler.size();
  auto &rng = sampler.particle().rng();
  auto weight = sampler.particle().weight().data();
  Vector<std::size_t> rep(N);
  Vector<std::size_t> idx(M);
  ResampleMultinomial resample;
  resample(M, N, rng, weight, rep.data());
  resample_trans_rep_index(M, N, rep.data(), idx.data());
  Particle<T> particle(M);
  for (std::size_t i = 0; i != M; ++i) {
      auto sp_dst = particle.sp(i);
      auto sp_src = sampler.partice().sp(idx[i]);
      // Assuming T is a subclass of StateMatrix
      for (std::size_t d = 0; d != sp_dst.dim(); ++d)
      sp_dst.state(d) = sp_src.state(d);
  }
  // Copy other data of class T if any
  sampler.particle() = std::move(particle);
\end{cppcode}
