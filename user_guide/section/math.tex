\section{Mathematical operations}
\label{sec:Mathemtical operations}

\subsection{Constants}
\label{sub:Constants}

\begin{table}[t]
  \begin{tabu}{X[2l]X[l]X[2l]X[l]X[2l]X[l]}
    \toprule
    Function & Value &
    Function & Value &
    Function & Value \\
    \midrule
    \texttt{pi}             & $\pi$              &
    \texttt{pi\_2}          & $2\pi$             &
    \texttt{pi\_inv}        & $1/\pi$            \\
    \texttt{pi\_sqr}        & $\pi^2$            &
    \texttt{pi\_by2}        & $\pi/2$            &
    \texttt{pi\_by3}        & $\pi/3$            \\
    \texttt{pi\_by4}        & $\pi/4$            &
    \texttt{pi\_by6}        & $\pi/6$            &
    \texttt{pi\_2by3}       & $2\pi/3$           \\
    \texttt{pi\_3by4}       & $3\pi/4$           &
    \texttt{pi\_4by3}       & $4\pi/3$           &
    \texttt{sqrt\_pi}       & $\sqrt{\pi}$       \\
    \texttt{sqrt\_pi\_2}    & $\sqrt{2\pi}$      &
    \texttt{sqrt\_pi\_inv}  & $\sqrt{1/\pi}$     &
    \texttt{sqrt\_pi\_by2}  & $\sqrt{\pi/2}$     \\
    \texttt{sqrt\_pi\_by3}  & $\sqrt{\pi/3}$     &
    \texttt{sqrt\_pi\_by4}  & $\sqrt{\pi/4}$     &
    \texttt{sqrt\_pi\_by6}  & $\sqrt{\pi/6}$     \\
    \texttt{sqrt\_pi\_2by3} & $\sqrt{2\pi/3}$    &
    \texttt{sqrt\_pi\_3by4} & $\sqrt{3\pi/4}$    &
    \texttt{sqrt\_pi\_4by3} & $\sqrt{4\pi/3}$    \\
    \texttt{ln\_pi}         & $\ln{\pi}$         &
    \texttt{ln\_pi\_2}      & $\ln{2\pi}$        &
    \texttt{ln\_pi\_inv}    & $\ln{1/\pi}$       \\
    \texttt{ln\_pi\_by2}    & $\ln{\pi/2}$       &
    \texttt{ln\_pi\_by3}    & $\ln{\pi/3}$       &
    \texttt{ln\_pi\_by4}    & $\ln{\pi/4}$       \\
    \texttt{ln\_pi\_by6}    & $\ln{\pi/6}$       &
    \texttt{ln\_pi\_2by3}   & $\ln{2\pi/3}$      &
    \texttt{ln\_pi\_3by4}   & $\ln{3\pi/4}$      \\
    \texttt{ln\_pi\_4by3}   & $\ln{4\pi/3}$      &
    \texttt{e}              & $\EE$              &
    \texttt{e\_inv}         & $1/\EE$            \\
    \texttt{sqrt\_e}        & $\sqrt{\EE}$       &
    \texttt{sqrt\_e\_inv}   & $\sqrt{1/\EE}$     &
    \texttt{sqrt\_2}        & $\sqrt{2}$         \\
    \texttt{sqrt\_3}        & $\sqrt{3}$         &
    \texttt{sqrt\_5}        & $\sqrt{5}$         &
    \texttt{sqrt\_10}       & $\sqrt{10}$        \\
    \texttt{sqrt\_1by2}     & $\sqrt{1/2}$       &
    \texttt{sqrt\_1by3}     & $\sqrt{1/3}$       &
    \texttt{sqrt\_1by5}     & $\sqrt{1/5}$       \\
    \texttt{sqrt\_1by10}    & $\sqrt{1/10}$      &
    \texttt{ln\_2}          & $\ln{2}$           &
    \texttt{ln\_3}          & $\ln{3}$           \\
    \texttt{ln\_5}          & $\ln{5}$           &
    \texttt{ln\_10}         & $\ln{10}$          &
    \texttt{ln\_inv\_2}     & $1/\ln{2}$         \\
    \texttt{ln\_inv\_3}     & $1/\ln{3}$         &
    \texttt{ln\_inv\_5}     & $1/\ln{5}$         &
    \texttt{ln\_inv\_10}    & $1/\ln{10}$        \\
    \texttt{ln\_ln\_2}      & $\ln\ln{2}$        &
    &                    &
    &                    \\
    \bottomrule
  \end{tabu}
  \caption{Mathematical constants. Note: All functions are prefixed by
    \cppinline{const_}.}
  \label{tab:Mathematical constants}
\end{table}

The library defines some mathematical constants in the form of
\cppinline{constexpr} functions. For example, to get the value of $\pi$ with a
desired precision, one can call the following.
\begin{cppcode}
  auto pi_f = const_pi<float>();
  auto pi_d = const_pi<double>();
  auto pi_l = const_pi<long double>();
\end{cppcode}
The compiler will evaluate these values at compile-time and thus there is no
performance difference from hard-coding the constants in the program, while the
readability is improved. All defined constants are listed in
table~\ref{tab:Mathematical constants}. Note that all functions has a prefix
\cppinline{const_}, which is omitted in the table.

\subsection{Vectorized operations}
\label{sub:Vectorized operations}

\begin{table}[t]
  \begin{tabu}{X[l]X[l]X[l]X[l]}
    \toprule
    Function & Operation & Function & Operation \\
    \midrule
    \texttt{add(n, a, b, y)}    & $y = a + b$                               &
    \texttt{sub(n, a, b, y)}    & $y = a - b$                               \\
    \texttt{sqr(n, a, y)}       & $y = a^2$                                 &
    \texttt{mul(n, a, b, y)}    & $y = ab$                                  \\
    \texttt{abs(n, a, y)}       & $y = |a|$                                 &
    \texttt{fma(n, a, b, c, y)} & $y = ab + c$                              \\
    \texttt{inv(n, a, y)}       & $y = 1 / a$                               &
    \texttt{div(n, a, b, y)}    & $y = a / b$                               \\
    \texttt{sqrt(n, a, y)}      & $y = \sqrt{a}$                            &
    \texttt{invsqrt(n, a, y)}   & $y = 1 / \sqrt{a}$                        \\
    \texttt{cbrt(n, a, y)}      & $y = \sqrt[3]{a}$                         &
    \texttt{invcbrt(n, a, y)}   & $y = 1 / \sqrt[3]{a}$                     \\
    \texttt{pow2o3(n, a, y)}    & $y = a^{2/3}$                             &
    \texttt{pow3o2(n, a, y)}    & $y = a^{3/2}$                             \\
    \texttt{pow(n, a, b, y)}    & $y = a^b$                                 &
    \texttt{hypot(n, a, b, y)}  & $y = \sqrt{a^2 + b^2}$                    \\
    \texttt{exp(n, a, y)}       & $y = \EE^a$                               &
    \texttt{exp2(n, a, y)}      & $y = 2^a$                                 \\
    \texttt{exp10(n, a, y)}     & $y = 10^a$                                &
    \texttt{expm1(n, a, y)}     & $y = \EE^a - 1$                           \\
    \texttt{log(n, a, y)}       & $y = \ln(a)$                              &
    \texttt{log2(n, a, y)}      & $y = \log_2(a)$                           \\
    \texttt{log10(n, a, y)}     & $y = \log_{10}(a)$                        &
    \texttt{log1p(n, a, y)}     & $y = \ln(a + 1)$                          \\
    \texttt{cos(n, a, y)}       & $y = \cos(a)$                             &
    \texttt{sin(n, a, y)}       & $y = \sin(a)$                             \\
    \texttt{sincos(n, a, y, z)} & $y = \sin(a)$, $z = \cos(a)$              &
    \texttt{tan(n, a, y)}       & $y = \tan(a)$                             \\
    \texttt{acos(n, a, y)}      & $y = \arccos(a)$                          &
    \texttt{asin(n, a, y)}      & $y = \arcsin(a)$                          \\
    \texttt{atan(n, a, y)}      & $y = \arctan(a)$                          &
    \texttt{acos(n, a, y)}      & $y = \arccos(a)$                          \\
    \texttt{atan2(n, a, b, y)}  & $y = \arctan(a / b)$                      &
    \texttt{cosh(n, a, y)}      & $y = \cosh(a)$                            \\
    \texttt{sinh(n, a, y)}      & $y = \sinh(a)$                            &
    \texttt{tanh(n, a, y)}      & $y = \tanh(a)$                            \\
    \texttt{acosh(n, a, y)}     & $y = \mathrm{arc}\cosh(a)$                &
    \texttt{asinh(n, a, y)}     & $y = \mathrm{arc}\sinh(a)$                \\
    \texttt{atanh(n, a, y)}     & $y = \mathrm{arc}\tanh(a)$                &
    \texttt{erf(n, a, y)}       & $y = \mathrm{erf}(a)$                     \\
    \texttt{erfc(n, a, y)}      & $y = \mathrm{erfc}(a)$                    &
    \texttt{cdfnorm(n, a, y)}   & $y = 1 - \mathrm{erfc}(a / \sqrt{2}) / 2$ \\
    \texttt{lgamma(n, a, y)}    & $y = \ln\Gamma(a)$                        &
    \texttt{tgamma(n, a, y)}    & $y = \Gamma(a)$                           \\
    \bottomrule
  \end{tabu}
  \caption{Vectorized mathematical operations}
  \label{tab:Vectorized mathematical operations}
\end{table}

The library provides a set of functions for vectorized mathematical operations.
For example,
\begin{cppcode}
  std::size_t n = 1000;
  vsmc::Vector<double> a(n);
  vsmc::Vector<double> b(n);
  vsmc::Vector<double> y(n);
  // Fill vectors a and b
  add(n, a.data(), b.data(), y.data());
\end{cppcode}
performs addition for vectors. If the input $a$ and $b$ are pointers to length
$n$ arrays, and the output $y$ is a pointer to array of the same length, then
this function call compute $y_i = a_i + b_i$ for $i=1,\dots,n$. Either $a$ and
$b$ can also be scalars. For example, if $b$ is a scalar, then the operation
performed is $y_i = a_i + b$. The functions defined are listed in
table~\ref{tab:Vectorized mathematical operations}. For each function, the
first parameter is always the length of the vector $n$, and the last is a
pointer to the output $y$ (except \cppinline{sincos} which has two output
pointers $y$ and $z$). For all functions, the output is always a vector. If
there are more than one input pointer, then some of them, but not all, can be
scalars.
